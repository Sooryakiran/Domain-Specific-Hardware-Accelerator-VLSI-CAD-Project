\contribution{Demo, Vector Negate}
\shortcontributor{CS6230 : CAD for VLSI Project Report}
\shortcontribution{Demo}
\headnum{10}
\begin{paper}
\renewcommand*{\pagemark}{}

\section*{}
In this demo, we will be generating a 32-bit CPU and attach it with a vector accelerator. \texttt{src\Demo\Demo2.bsv} contains the required setup concerning this. 
\section*{Looking at Demo2.bsv\sdot}
\texttt{Demo2.bsv} is different from \texttt{Demo1.bsv}. It contains our vector accelerator attached to the Bus.
\begin{minted}[
bgcolor=Gray !5,
breaklines
]{haskell}
`include <VX_Address.bsv> // Location where Accelerator is memory mapped

`define WORD_LENGTH 32   // Here we are generating a 32 bit CPU
`define DATA_LENGTH 32
`define BUS_DATA_LEN 128 // When changing bus width, remember to increase memory ports 
`define ADDR_LENGTH 20
`define VECTOR_DATA_SIZE `BUS_DATA_LEN
`define VX_STORAGE_SIZE 2

`define GRANULARITY 8    // Smallest addressible unit (1 byte)
`define RAM_BYTES 64     // Ram size (number of addressible units)
`define RAM_PORTS 16     // 16 ports, 1 byte per port for 128 bit bus
`define RAM_ADDRESS_OFFSET 1024

`define CONSOLE_ADDRESS 128
\end{minted}\\\\
\nointed The vector accelerator is defined as attached as follows,
\begin{minted}[
bgcolor=Gray !5,
breaklines
]{haskell}
VectorUnary #(`DATA_LENGTH,
               `VECTOR_DATA_SIZE,
               `BUS_DATA_LEN,
               `ADDR_LENGTH,
               `GRANULARITY) vec_Unary <- mkVectorUnary (`VX_ADDRESS, `VX_STORAGE_SIZE, 7);
                                        // VX_ADDRESS <- Memory mapped address of Accelerator
                                        // VX_STORAGE_SIZE <- Depth of temporary storage FIFOs
Vector #(2, BusMaster #(`BUS_DATA_LEN, 
                        `ADDR_LENGTH, 
                        `GRANULARITY)) master_vec;

Vector #(3, BusSlave  #(`BUS_DATA_LEN, 
                        `ADDR_LENGTH, 
                        `GRANULARITY)) slave_vec;

...
...
...

slave_vec[2]  = vec_Unary.bus_slave;

Bus #(2, 3, `BUS_DATA_LEN, 
            `ADDR_LENGTH, 
            `GRANULARITY) bus <- mkBus(master_vec, slave_vec);

mkConnection (master_vec, bus);
mkConnection (slave_vec, bus);

\end{minted}\\\\
\section*{Running Demo Files\sdot}
Four separate assembly files are provided for demonstrating vector negation on four different datatypes, \texttt{int8, int16, int32 & float32}. These files are located at,
\begin{minted}[
bgcolor=Gray !5,
breaklines
]{text}
src/asm/vec_neg_f32_demo.asm
src/asm/vec_neg_i8_demo.asm
src/asm/vec_neg_i16_demo.asm
src/asm/vec_neg_i32_demo.asm
\end{minted}\\\\
\nointend Here is a snippet of \texttt{src/Demo/vec\_neg\_i32\_demo.asm}. It initializes a vector of \texttt{int\_32}, of length 10 with numbers from 0 to 9. Vector negation is done and the output is printed. 

\begin{minted}[
bgcolor=Gray !5,
breaklines,
escapeinside=||
]{asm}
ASG_32 R1 0         ; Initialisation
ASG_32 R2 10        ; Num loops
ASG_32 R3 1024      ; Address
ASG_32 R4 1         ; Value increment delta
ASG_32 R5 4         ; Address increment delta
ASG_32 R0 128       ; Address of console
loop_init: STORE_32 R1 R3      ;
    STORE_32 R1 R0      ; Print console
    ADD_I32 R1 R4 R1    ; Increment value
    ADD_I32 R3 R5 R3    ; Increment address
    ASG_32 R7 1
    IS_EQ R1 R2 R6      ; Compare
    ADD_I32 R6 R7 R6
    ASG_32 R7 |\$|loop_init
    JMPIF R7 R6         ; Jump if not Eq
ASG_32 R3 1024
VEC_NEG_I32 R3 R2 R3    ; Vector negation
ASG_32 R1 0
loop_print: LOAD_32 R3 R4   ; Load from memory
    STORE_32 R4 R0          ; Print
    ADD_I32 R3 R5 R3        ; Incement address
    ASG_32 R4 1
    ADD_I32 R1 R4 R1        ; Increment Loop
    ASG_32 R7 1
    IS_EQ R1 R2 R6          ; Compare
    ADD_I32 R6 R7 R6
    ASG_32 R7 |\$|loop_print
    JMPIF R7 R6             ; Jump if not Eq
\end{minted}\\\\

\section*{Running the Simulation\sdot}
Generate machine code by running, 
\begin{minted}[
bgcolor=Gray !5,
breaklines
]{bash}
cd /src/asm
./asm vec_neg_i32_demo.asm -o vector
\end{minted}\\\\
\nointend Since, we are generating for a 32-bit CPU in this example, we do not have to specify the \texttt{-w N} option. 32-bit is the default value. Also note that the output file name is consistent with the input taken from \texttt{Demo2.bsv}\\\\
\nointend If compiling \texttt{Demo2.bsv} for the first time, run,
\begin{minted}[
bgcolor=Gray !5,
breaklines
]{bash}
cd src/Demo/
./compile_and_sim.sh Demo2.bsv 
\end{minted}\\\\
\nointend Else, if you have already compiled once, then run,
\begin{minted}[
bgcolor=Gray !5,
breaklines
]{bash}
cd src/Demo/
./out
\end{minted}\\\\
\nointend If no errors occur, you will get output similar to as shown below. Note that if you are running the demo on \texttt{float32}, the outputs are in IEEE 754 format hex and appear to not convey any legible information in \texttt{int}.
\begin{minted}[
bgcolor=Gray !5,
breaklines
]{bash}
CONSOLE 00000000 |           0
CONSOLE 00000001 |           1
CONSOLE 00000002 |           2
CONSOLE 00000003 |           3
CONSOLE 00000004 |           4
CONSOLE 00000005 |           5
CONSOLE 00000006 |           6
CONSOLE 00000007 |           7
CONSOLE 00000008 |           8
CONSOLE 00000009 |           9
CONSOLE 00000000 |           0
CONSOLE ffffffff |          -1
CONSOLE fffffffe |          -2
CONSOLE fffffffd |          -3
CONSOLE fffffffc |          -4
CONSOLE fffffffb |          -5
CONSOLE fffffffa |          -6
CONSOLE fffffff9 |          -7
CONSOLE fffffff8 |          -8
CONSOLE fffffff7 |          -9

\end{minted}\\

\end{paper}

