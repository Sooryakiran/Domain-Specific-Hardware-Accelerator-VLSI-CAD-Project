\contribution{Demo, Statistics Minima}
\shortcontributor{CS6230 : CAD for VLSI Project Report}
\shortcontribution{Demo}
\headnum{11}
\begin{paper}
\renewcommand*{\pagemark}{}

\section*{}
In this demo, we will initialize a vector with some values, and we will print its minima and minimum index. The assembly files required for this demo are located at,
\begin{minted}[
bgcolor=Gray !5,
breaklines
]{text}
src/asm/vec_min_f32_demo.asm
src/asm/vec_min_i8_demo.asm
src/asm/vec_min_i16_demo.asm
src/asm/vec_min_i32_demo.asm
\end{minted}\\\\
\nointend Here is a snippet from \texttt{vec\_min\_i8\_demo.asm}
\begin{minted}[
bgcolor=Gray !5,
breaklines,
escapeinside=||
]{asm}
NOP                 ; Store Array [13, 12, 11, -13, 15]
ASG_32 R3 128       ; Console address
ASG_8 R1 13         ; Value
ASG_32 R2 1024      ; Address
STORE_8 R1 R2       ; Store to RAM
STORE_8 R1 R3       ; Print
ASG_8 R1 12         ; Value
ASG_32 R2 1025      ; Address
STORE_8 R1 R2       ; Store to RAM
STORE_8 R1 R3       ; Print
ASG_8 R1 11         ; Value
ASG_32 R2 1026      ; Address
STORE_8 R1 R2       ; Store to RAM
STORE_8 R1 R3       ; Print
ASG_8 R1 -13        ; Value
ASG_32 R2 1027      ; Address
STORE_8 R1 R2       ; Store to RAM
STORE_8 R1 R3       ; Print
ASG_8 R1 15         ; Value
ASG_32 R2 1028      ; Address
STORE_8 R1 R2       ; Store to RAM
STORE_8 R1 R3       ; Print
NOP
ASG_32 R1 1024      ; Vector starting location
ASG_32 R2 5         ; Vector size
ASG_32 R4 1024      ; Minimum dst
ASG_32 R5 1025      ; Argmin dst
VEC_MIN_I8 R1 R2 R4 R5  ; Vec op
LOAD_8 R4 R6        ; Load minima
STORE_8 R6 R3       ; Print minima
LOAD_8 R5 R7        ; Load argmin
STORE_8 R7 R3       ; Print argmin
\end{minted}\\\\

\section*{Running the Simulation\sdot}
Generate machine code by running, 
\begin{minted}[
bgcolor=Gray !5,
breaklines
]{bash}
cd /src/asm
./asm vec_min_i8_demo.asm -o vector
\end{minted}\\\\
\nointend Since, we are generating for a 32-bit CPU in this example, we do not have to specify the \texttt{-w N} option. 32-bit is the default value. Also note that the output file name is consistent with the input taken from \texttt{Demo2.bsv}\\\\
\nointend If compiling \texttt{Demo2.bsv} for the first time, run,
\begin{minted}[
bgcolor=Gray !5,
breaklines
]{bash}
cd src/Demo/
./compile_and_sim.sh Demo2.bsv 
\end{minted}\\\\
\nointend Else, if you have already compiled once, then run,
\begin{minted}[
bgcolor=Gray !5,
breaklines
]{bash}
cd src/Demo/
./out
\end{minted}\\\\
\nointend If no errors occur, you will get output similar to as shown below. Note that if you are running the demo on \texttt{float32}, the outputs are in IEEE 754 format hex and appear to not convey any legible information in \texttt{int}. The last 2 lines represent the minimum value and the index of the minimum value (starting from 0) respectively. If multiple minima exists, the index with the lower value is outputted.
\begin{minted}[
bgcolor=Gray !5,
breaklines
]{bash}
CONSOLE 0d |   13
CONSOLE 0c |   12
CONSOLE 0b |   11
CONSOLE f3 |  -13
CONSOLE 0f |   15
CONSOLE f3 |  -13
CONSOLE 03 |    3
\end{minted}\\
\end{paper}