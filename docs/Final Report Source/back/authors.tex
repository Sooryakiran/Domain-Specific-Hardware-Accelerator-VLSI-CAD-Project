% add Authors section as a \part in the ToC
\addcontentsline{toc}{part}{Authors}
% use "authors" style for headers and footers
\pagestyle{authors}
% format "Authors" section as a new \chapter
\chapter*{Authors}
% pagenr at the bottom
\protect\thispagestyle{chaptertitlepage}

% Short Bios of authors start here
% Each author gets 1 \paragraph
% Each \paragraph starts with author name
% This author name is formatted as the \paragraph title to make it pop out more (bold and some space)


\paragraph{Elli Bleeker} is a postdoctoral researcher in the Research and Development Team at the Humanities Cluster, part of the Royal Netherlands Academy of Arts and Sciences. She specializes in digital scholarly editing and computational philology, with a focus on modern manuscripts and genetic criticism. Elli completed her PhD in 2017 at the University of Antwerp on the role of the scholarly editor in the digital environment. During her Early Stage Research Fellowship in the Marie Skłodowska-Curie funded DiXiT network (2014–2017), she received advanced training in manuscript studies, text modeling, and XML technologies. She has participated in the organization and teaching of workshops on scholarly editing with a focus on knowledge transfer and the application of computational methods in a humanities environment. She is also the Associate Editor of Variants from this issue onwards.

\paragraph{Wout Dillen} holds a PhD in Literature with a focus on text encoding and digital scholarly editing from the University of Antwerp. From 2016 to 2017, Wout held a Marie Skłodowska-Curie Experienced Research Fellowship in the Digital Scholarly Editions Initial Training Network (DiXiT ITN) at the University of Borås. Since 2016 he has been the Coordinator of platform{DH} and the Coordinator of the University of Antwerp’s contribution to DARIAH-Flanders. Wout currently serves as the Secretary of the European Society of Textual Scholarship (ESTS), as a member of the Steering Committee of the DH Benelux. For the current issue, Wout is an Associate editor of Variants, and will be General Editor from issue 15 onwards. Besides these, he is also on the editorial boards of the Review Journal for Digital Editions and Resources (RIDE) and the upcoming DH Benelux journal.

\paragraph{Aodhán Kelly} is an affiliated researcher in the Centre for Manuscript Genetics at the University of Antwerp and was a Marie Skłodowska-Curie Early Stage Research Fellow within the DiXiT network. Aodhán researches methods and models for the dissemination of digital scholarly editions to wider audiences and he defended a PhD thesis on this topic at the University of Antwerp in July 2017 with the title ``Disseminating digital scholarly editions of textual cultural heritage''.

\paragraph{Merisa Martinez} is a PhD Candidate in the Swedish School of Library and Information Science at the University of Borås, a Visiting Research Fellow at the Cambridge Digital Library, and a member of the Program Committee for the 2019 Digital Access to Textual Cultural Heritage (DaTech) Conference. From 2014 to 2017, she held a Marie Skłodowska-Curie Early Stage Research Fellowship in the Digital Scholarly Editions Initial Training Network (DiXiT ITN), where she also served for three years as an elected Student Representative to the Project Advisory Board. Merisa is currently writing a doctoral dissertation on the intersection of digital textual scholarship and the digitization process in libraries, which she will defend in 2019.

\paragraph{Anna-Maria Sichani} is a Research Fellow in Media History and Historical Data Modelling on the AHRC-funded ‘Connected Histories of the BBC’ project at the Department of Media, Film and Music at University of Sussex and Sussex Humanities Lab. In 2018, she completed her PhD in Modern Greek Philology and Cultural Studies at University of Ioannina in Greece. From 2017 to 2018, Anna-Maria was the Communications Fellow for the Alliance of Digital Humanities Organizations (ADHO). Previously, Anna-Maria held a Marie Skłodowska-Curie Early Stage Research Fellowship in the Digital Scholarly Editions Initial Training Network (DiXiT ITN) at Huygens ING and a PhD Research Fellowship at King’s College Digital Lab in London. She has collaborated on a number of Digital Humanities projects including the COST Action “Distant Reading for European Literary History," Transcribe Bentham, and DARIAH. Currently, Anna-Maria serves on the Editorial Board of The Review Journal for Digital Editions and Resources (RIDE) and at The Programming Historian.